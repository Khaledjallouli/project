%!TEX root = ../../main.tex

\chapter{Introduction}

The project is about predicting the outcome of soccer games and training a neural network. For the initial training the database from Kaggle in source \cite{kggl:2019} was used. 


\section{Motivation}
Machine Learning with Neural Networks is a main topic in the software development field. Many companies start to try analyze different kinds of data with this approach. Because of this reason as master students in information systems it is very interesting to gather knowledge about machine learning and neural networks. For this matter a prediction of soccer matches is a very nice procedure to do. This topic does not require to gather much additional knowledge about what you need for analyzing the data and preparing them for prediction. Additionally there is a free database with data of matches for a lot of seasons. In the beginning, the knowledge about machine learning is very low and the goal is to improve the handling of python in combination with machine learning techniques. This includes pre-processing data with Pandas, normalizing data sets and extracting the right features, as well as developing adequate models for the machine-learning algorithm. Additionally to gaining knowledge an algorithm will be created, which predicts the outcome of soccer games with a decent accuracy.
\section{Goals}
The main goals of this project can be divided into 2 parts:
\begin{itemize}
	\item \textbf{Gaining knowledge}
		\begin{itemize}
			\item Improve skills in python\newline
			For the the project and for future work with neural networks it is necessary to become familiar with python. There are many things which are different in python than in other programming languages. All these things have to be discovered and it is necessary to learn how to deal with the python syntax. Additionally it is important to gain knowledge about some library, which you can use with python and make it easier to solve some problems.
			\item Gaining knowledge about data pre-processing\newline 
			It is very important to edit the data in a way that it is ready for a neural network. Algorithms for example can not deal with strings, only with numbers. For this matter, there has to be some knowledge gained, before the data can be processed in a proper way.
			\item Gaining knowledge about neural networks\newline
			Neural networks are one of the most famous topics in the technical world these times. So as master students in computer science it is urgent to gain some knowledge about neural networks and machine learning. For the project it is necessary to understand the overall technology of the whole topic, too.
		\end{itemize}
	\item \textbf{Outcome of the project}
		\begin{itemize}
			\item Finding the right features\newline
			As a first step a feature selection should be done. For this, the database has to be analyzed and there some first features should be chosen by gut feeling. These features have to be checked whether they are independent of each other. During the project it should always be reconsidered, if it makes sense to add some more features and checked whether it improves the accuracy.
			\item Normalizing the features in a proper way\newline
			The features have to be normalized before using them in a prediction model. For this procedure, it is necessary to find fitting algorithms or write some. The normalization of the data is a major step in the project development.
			\item Finding a good model for the prediction\newline
			There are different libraries available for machine learning. The recommended library is Tensorflow in combination with Keras. Additionally there has to be research for other libraries and approaches. The different models are then compared and in the end the best model will be chosen. It is more important to look at the prediction of the actual winner than the prediction of the goals in a different way. So this step has to be done for both ideas.
			\item Getting a decent accuracy with the prediction\newline
			The accuracy for the winner should be higher than 50\% in the end. With more than 50\% you would theoretically always be earning money. The main goal is to improve the accuracy during the whole project. For the predicting of the goals the accuracy can be lower than 50\%, because it is harder to predict the exact number of goals which were shot by each team.
			\item Creating a Website\newline
			To make it easy for people to look through the former predictions and predict upcoming soccer games, a website has to be implemented. For this matter it is necessary to create a frontend for showing the data in an easy to understand and nice way, a backend for data storage and processing and an API to connect both.
		\end{itemize}
\end{itemize}

The first project team had the members Khaled Jallouli, Martin Schmidt, Sergej Dechant and Lisa Boos. The second team contained Khaled Jallouli and Lisa Boos from the first team and the two new team members Hemlata Prajapati and Till Hoffmann. The knowledge gaining part was for both of the teams of major interest, except of the `Gaining knowledge in data pre-processing', this part was mostly already done by the first team. In the outcome of the project, `Finding the right features' and 'Normalizing the features' was only done by the former team. The `Finding a good model for prediction' was done by the first team for the winner prediction and for the goal prediction by the new team. The quality criteria was developed only by the new team as well as the `Creating a Website' part.

\section{Procedure}

For the procedure process it was decided to follow the CRISP-DM model. Image \ref{CRISPImage} shows the different stages of CRISP-DM.

\begin{figure}[h]
\centering
\includegraphics[width=0.5\textwidth]{images/CRISP-DM_Process_Diagram1.png}
\caption{The six stages of CRISP-DM \cite{CRISPDig:2012}}
\label{CRISPImage}
\end{figure}

\begin{itemize}
\item Business Understanding \newline
As a first step the goals of the project have to be set, which are already described in the chapter before. It is necessary to clearify what exactly is the required outcome. For this matter in this project SCRUM was used to organize the project. The teams met once per week to discuss the achievements from each single person. All three to four weeks there was a sprint review with the project owners where the group achievements were discussed and if the project still leads to the right direction. In the initial sprint meeting the common parameters of the project were set.
\item Data Understanding \newline
Before starting to create a model or select features the dataset has to be understood. For this matter it is necessary to understand the structure of the set including the different columns and what they indicate. Additionally, it is important how many datasets are available and which types of value it they contain. It is also important to learn something about soccer and what are the important factors of a match.
\item Data Preparation \newline
Before it is possible to use the Dataset in a neuronal network it is necessary to prepare the data. For this matter as first step there has to be a feature selection. So one has to decide which columns are important for the later prediction. After dropping the unnecessary columns it is important to prepare the resulting dataset. The records which have empty attributes have to be deleted or filled with specific values. Additionally the data has to be sorted and aggregated. At the end it has to be split into test and training data. 
\item Modeling \newline
The resulting dataset from the step before is ready to be used for different models. In this step different models will be prepared to find a decent one. Additionally it is necessary to define a measurable goal how good the model can get.
\item Evaluation \newline
In the end a evaluation of the accomplished goals will be made and whether it is possible to increase the accuracy any further.  
\item Deployment \newline
As the idea of the project is to have a product to show off at the end, this has to be implemented along the way. The finished product should have the ability to predict matches, using the beforehand implemented model(s). It should also list historic games with the prediction of the model, so the user is able to compare prediction versus reality.

\end{itemize}

The steps `Data Preparation' and `Modeling' will repeat as long as the resulting accuracy is not getting any better or the resulting one is satisfying. For all the steps it is necessary to do a lot of research to find the right techniques to do the tasks in a proper way.  \newline
To be able deploy a product after the 'Data Preparation' and 'Modeling' loop is finished, an architecture and fitting software has to be developed, which makes use of all the previous steps.
