%!TEX root = ../../main.tex

\chapter{Conclusion}
\label{chap:conclusion}
All main goals for the project were reached. Source code of the project is available in the following GitHub repository (\url{https://github.com/Khaledjallouli/project}). Both approaches have a good quality and the authors have learned a lot about machine learning. By organising with SCRUM, it was possible to delegate the tasks in a way that the best end solution was achieved. Additionally, by working with SCRUM the authors can use their experiences in their further career.

The project is properly documented, so that a future team is able to continue with our work on the project. By implementing a deployable version of the software, we also enable future teams to continue with a stable codebase, without any flourishes. The biggest issues were both right at the beginning and at the middle of the project: At first, it was not known how to reach the goals or even what the goals actually are. In the midterm, it was realized some mistakes were made on the way and a lot time was used on researching on how to fix them.

For the future it would be awesome if the communication between the product owners and the students would be better. Not only on how the project should start and what the main goals are, but also how the grade will be calculated and how this report should be structured. Perhaps it would be a good start to create a one- or two-pages summary with all the common guidelines the students can use to orient themselves.

By developing a new quality criteria, the result is now very close to how real world betting works. Because the quality of both approaches with this new criteria is above 70\%, it would be interesting so use the prediction for betting on matches and see whether money could be won. Especially because the betting odds are input data for the model. It would be great to remove this dependency, so outcomes of matches can be predicted regardless of any bettings odds and therefore regardless of any external (changing) data.
On the other side, the actual accuracy is only hardly over a 50\% value. This means there is still room to improve the models.
